% Template:     Informe/Reporte LaTeX
% Documento:    Archivo principal
% Versión:      6.0.3 (27/10/2018)
% Codificación: UTF-8
%
% Autor: Pablo Pizarro R. @ppizarror
%        Facultad de Ciencias Físicas y Matemáticas
%        Universidad de Chile
%        pablo.pizarro@ing.uchile.cl, ppizarror.com
%
% Manual template: [https://latex.ppizarror.com/Template-Informe/]
% Licencia MIT:    [https://opensource.org/licenses/MIT/]

% CREACIÓN DEL DOCUMENTO
\documentclass[letterpaper,11pt]{article} % Articulo tamaño carta, 11pt
\usepackage[utf8]{inputenc} % Codificación UTF-8

% INFORMACIÓN DEL DOCUMENTO
\def\titulodelinforme {Laboratorio N°2}
\def\temaatratar {Oscilaciones forzadas}

\def\autordeldocumento {Cristopher Gómez}
\def\nombredelcurso {Vibraciones Mecánicas}
\def\codigodelcurso {ME4701}

\def\nombreuniversidad {Universidad de Chile}
\def\nombrefacultad {Facultad de Ciencias Físicas y Matemáticas}
\def\departamentouniversidad {Departamento de Ingeniería Mecánica}
\def\imagendepartamento {departamentos/dimec}
\def\imagendepartamentoescala {0.2}
\def\localizacionuniversidad {Santiago, Chile}

% INTEGRANTES, PROFESORES Y FECHAS
\def\tablaintegrantes {
\begin{tabular}{ll}
	Integrantes:
	& \begin{tabular}[t]{@{}l@{}}
		Cristopher Gómez\\
		Grupo 2
	\end{tabular} \\
	Profesor:
	& \begin{tabular}[t]{@{}l@{}}
		Karim Pichara
	\end{tabular} \\
	Auxiliar:
	& \begin{tabular}[t]{@{}l@{}}
		Tomás Nahum
	\end{tabular} \\
	Ayudantes:
	& \begin{tabular}[t]{@{}l@{}}
		Javier Videla \\
		Maximiliano Barra \\
		Nicolás Gajardo \\
		Pablo Pogorelow
	\end{tabular} \\
	\multicolumn{2}{l}{Fecha de entrega: \today} \\
	\multicolumn{2}{l}{\localizacionuniversidad}
\end{tabular}}{
}

% CONFIGURACIONES
\input{lib/config}

% IMPORTACIÓN DE LIBRERÍAS
\input{lib/env/imports}

% IMPORTACIÓN DE FUNCIONES Y ENTORNOS
\input{lib/cmd/all}

% IMPORTACIÓN DE ESTILOS
\input{lib/style/all}

% CONFIGURACIÓN INICIAL DEL DOCUMENTO
\input{lib/cfg/init}

% INICIO DE LAS PÁGINAS
\begin{document}

% PORTADA
\input{lib/page/portrait} % Se puede borrar

% CONFIGURACIÓN DE PÁGINA Y ENCABEZADOS
\input{lib/cfg/page}

% RESUMEN O ABSTRACT

% T

% CONFIGURACIONES FINALES
\input{lib/cfg/final}

% ======================= INICIO DEL DOCUMENTO =======================
 % Ejemplo, se puede borrar
\section{Introducción}
Las vibraciones mecánicas que se encuentran normalmente en la industria provienen de un sistema que se encuentra forzado a oscilar por una fuerza externa. El ejemplo común es el de una máquina con un elemento rotatorio que genera vibraciones producto de un desbalance en su distribución de masa. Debido a que el mecanismo ubicuo de transformación de energía eléctrica o química a energía mecánica es a través de elementos rotatorios, este fenómeno de oscilaciones forzadas afecta a un gran número de industrias. 
Un punto importante es la existencia del fenómeno de resonancia que ocurre cuando la fuerza externa tiene una frecuencia cercana a la frecuencia natural del sistema donde actúa. En resonancia, el sistema oscila con amplitud máxima. Esto puede causar una falla mecánica rápidamente por lo que es de suma importancia mantener controlado el sistema para evitar que entre en resonancia. 

\section{Descripción del problema}
El sistema experimental (Figura \ref{weaita}) consiste en una barra sólida anclada con un pivote a una estructura inmóvil.  En el otro extremo la barra se une a un resorte. Adicionalmente, una masa se encuentra anclada al extremo derecho de la barra. Un motor desbalanceado esta montado en el punto central de la barra.

Los parámetros del sistema se encuentran en la Tabla \ref{tab:tablita}.

\insertimage[\label{weaita}]{test.png}{width=6cm}{Sistema mecánico a estudiar.}
\begin{table}[hbt]
    \centering
    \begin{tabular}{c|c}
         Largo de la barra $[m]$&  $0.85$\\
         Sección viga $[mm^2]$&$25.5\cdot12.7$ \\
         Módulo de Young $[Pa]$& $2.1\cdot10^{11}$\\
         Densidad $[\frac{kg}{m^3}]$&$7800$ \\
         Rigidez del resorte $[\frac{N}{m}]$& $2500$\\
         Masa del motor $[kg]$&$5.105$  \\
     	 Masa de desbalance $[kg]$&$0.058$\\
     	 Radio de desbalance $[m]$&$4.5\cdot10^{-2}$\\
     	 Masa extra $[kg]$&$0.473$
    \end{tabular}
    \caption{Propiedades del sistema experimental.}
    \label{tab:tablita}
\end{table}

La velocidad del motor puede ser controlada para estudiar la respuesta del sistema. Se realizan 9 experiencias variando la velocidad del motor. En cada experiencia se mide el tiempo y la aceleración en la punta de la barra a través de un acelerómetro.

\section{Antecedentes}
La respuesta del sistema se obtiene a través del método de Lagrange asumiendo pequeñas oscilaciones. Se agrega un coeficiente de roce viscoso para modelar el roce que existe por la existencia de aire y por roces entre los elementos del sistema. El resultado del motor desbalanceado es una fuerza armónica como elemento forzante en el sistema. 
La solución es régimen estacionario corresponde a la Ecuación \ref{eq:xp}. La variable $X$ corresponde a la amplitud de la oscilación. La Ecuación \ref{eq:ap} entrega una relación entre la amplitud de la oscilación y la amplitud de la aceleración del sistema.
Para los cálculos se va a considerar $c=0$ y por lo tanto $\xi=0$ ya que no existe una fuente considerable de roce viscoso.  La Ecuación \ref{eq:X} permite calcular la amplitud de oscilación que es la variable a analizar en este trabajo.  Se define $r=\frac{w_n}{w_r}$.
\begin{equation}
M_{total} =\left(\frac{1}{3}M_{barra}+\frac{1}{4}M_{motor}+m\right)
\label{eq:respuesta}
\end{equation}
\begin{equation}
    M_{total}\ddot{x} + k\dot{x} + c\dot{x}= m_{0}e\omega_{r}^2\sin(\omega_{r}t)
    \label{eq:mtotal}
\end{equation}
\begin{equation}
x_p(t)=X\sin(\omega_rt-\theta)
\label{eq:xp}	
\end{equation}
\begin{equation}
\ddot{x}=-X\omega_{r}^2\sin(\omega_rt-\theta)
\label{eq:ap}
\end{equation}
\begin{equation}
	X = \frac{m_{0}e}{M_{total}}\frac{r^2}{\sqrt{(1-r^2)^2+(2\xi r)^2}}
	\label{eq:X}
\end{equation}
\begin{equation}
	\theta = \tan[-1](\frac{2\xi r}{1-r^2})
	\label{eq:teta}
\end{equation}
\section{Resultados}
\begin{images}[\label{imagenmultiple}]{Ejemplo de imagen múltiple.}
	\addimage{frecuencias3.png}{width=7cm}{Tranformada de Fourier para experimento 1}
	\addimage{frecuencias3-5.png}{width=7cm}{Transformada de Fourier para experimento}
	\addimage{frecuencias4.png}{width=7cm}{Transformada de Fourier para experimento 3}
	\addimage{frecuencias4-5.png}{width=7cm}{Transformada de Fourier para experimento 3}
	\addimage{frecuencias5.png}{width=7cm}{Transformada de Fourier para experimento 3}
	\addimage{frecuencias5-5.png}{width=7cm}{Transformada de Fourier para experimento 3}
\end{images}

\newpage
\begin{images}[\label{imagenmultiple}]{Ejemplo de imagen múltiple.}
\addimage{frecuencias6.png}{width=7cm}{Transformada de Fourier para experimento 3}
\addimage{frecuencias6-5.png}{width=7cm}{Transformada de Fourier para experimento 3}
\addimage{frecuencias7.png}{width=7cm}{Transformada de Fourier para experimento 3}
\end{images}
\begin{images}[\label{imagenmultiple}]{Ejemplo de imagen múltiple.}
	\addimage{respuesta5.png}{width=7cm}{Transformada de Fourier para experimento 3}
	\addimage{respuesta6.png}{width=7cm}{Transformada de Fourier para experimento 3}
\end{images}
\newpage
\begin{table}[hbt]
    \centering
    \begin{tabular}{c|c|c|c|c|c|c|c|c|c}
        Vel.&$3$&$3.5$&$4$&$4.5$&$5$&$5.5$&$6$&$6.5$&$7$\\
        \hline
          Exp.&$0.088$&$1.95$&1.38&1.20&18.04&4.85&-&1.46&1.38\\
          Teo.&$0.203$&$0.467$&$1.10$&$3.15$&$37.8$&$8.39$&$4.51$&$5.25$&$4.31$
         
    \end{tabular}
    \caption{Amplitud de aceleraciones $[\frac{m}{s^2}]$ teóricas y experimentales para cada velocidad de motor.}
    \label{tab:ffff}
\end{table}
\section{Análisis de resultados y conclusiones}
En la Tabla \ref{tab:ffff} se aprecia que los resultados experimentales son cercanos a los predichos por la teoría.  Sin embargo, se observa que para las velocidades bajas el error de predicción es menor que para las velocidades altas.

Se observa que la resonancia ocurre cerca de los $5 Hz$ lo que se corresponde al cálculo teórico de la frecuencia natural del sistema $w_n=31.8468 \frac{rad}{s}$ lo que es equivalente a $5.0686 Hz$.  Sin embargo, la aceleración predicha es casi el doble a la obtenida experimentalmente.  Esto se explica por varios factores: la existencia de errores en el montaje del sistema, la existencia de roce que disipa energía, errores del acelerómetro y el hecho de que en el modelo teórico se asume que el sistema se encuentra en régimen permanente. 
% FIN DEL DOCUMENTO
\end{document}
